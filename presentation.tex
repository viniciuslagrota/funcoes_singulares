\documentclass[mathserif]{beamer}

%\usetheme{LCOM}
\usetheme{Madrid}
\usepackage{hyperref}
\usepackage{listings}
\usepackage{graphicx}
\usepackage[portuguese]{babel}
\usepackage[utf8]{inputenc}
\usepackage[T1]{fontenc}
\usepackage{booktabs}
\usepackage{amsmath,amssymb,amsthm,mathrsfs,amsfonts,dsfont}
\usepackage{adjustbox}
\usepackage{array}
\usepackage{longtable}
\usepackage{psfrag}
\usepackage{caption}
\usepackage{placeins}
\usepackage{subfig}
\usepackage{multirow}
\usepackage[siunitx, american]{circuitikz}

\uselanguage{portuguese}
\languagepath{portuguese}
\deftranslation[to=portuguese]{Theorem}{Teorema}
\deftranslation[to=portuguese]{theorem}{teorema}

\definecolor{green_variavel}{rgb}{0.13,0.57,0.2}
\usecolortheme[named=green_variavel]{structure}

\newcommand{\wait}{\vfill}
%\newcommand{\wait}{\pause}
	
\DeclareMathAlphabet\mathbfcal{OMS}{cmsy}{b}{n}
\newcommand\blfootnote[1]{%
	\begingroup
	\renewcommand\thefootnote{}\footnote{#1}%
	\addtocounter{footnote}{-1}%
	\endgroup
}

\setbeamertemplate{frametitle continuation}[from second][]

\title[Funções Singulares]{Funções Singulares e suas aplicações} 
\author[Vinícius Lagrota]{Vinícius Lagrota Rodrigues da Costa}
\institute[CES]{Centro de Ensino Superior de Juiz de Fora}
\date{\today}

\begin{document}

\begin{frame}
\maketitle 
\end{frame}

\begin{frame}
\frametitle{Sumário}
\tableofcontents
\end{frame}

%------------------------------------------------------------------------------------
\section{Recapitulação}
%------------------------------------------------------------------------------------
\begin{frame}
\frametitle{Recapitulação}
	\begin{block}{Resistores}
			\begin{minipage}[b]{0.4\linewidth}
				\begin{center}
					\begin{circuitikz}[european voltages]
						%				\draw (0,0) to [I, l^=$i(t)$] (0,2);
						\draw [-latex] (0.6, 1.85) -- (1.25, 1.85);
						\draw node[] at (0.25, 1.75) {$i(t)$};
						\draw (0,2) -- (1.5,2);
						\draw (0,0) -- (1.5,0);
						\draw (1.5,0) to [R, l^= $R$, v_=$e(t)$] (1.5,2);
					\end{circuitikz}
				\end{center}
			\end{minipage}
			\begin{minipage}[b]{0.4\linewidth}
				\vfill
				\begin{equation}\label{key}
				e(t) = Ri(t)
				\end{equation}
				\vfill
				\begin{equation}\label{key}
				i(t) = \frac{1}{R}e(t)
				\end{equation}
				\vfill
			\end{minipage}
	\end{block}
	\begin{minipage}[b]{0.2\linewidth}
		\begin{center}
			\begin{circuitikz}[european voltages]
%				\draw (0,0) to [I, l^=$i(t)$] (0,2);
				\draw [-latex] (0.6, 1.85) -- (1.25, 1.85);
				\draw node[] at (0.25, 1.75) {$i(t)$};
				\draw (0,2) -- (1.5,2);
				\draw (0,0) -- (1.5,0);
				\draw (1.5,0) to [R, l^= $R$, v_=$e(t)$] (1.5,2);
			\end{circuitikz}
		\end{center}
	\end{minipage}
	\begin{minipage}[b]{0.7\linewidth}
		\begin{equation}\label{key}
			e(t) = Ri(t)
		\end{equation}
		\begin{equation}\label{key}
			i(t) = \frac{1}{R}e(t)
		\end{equation}
	\end{minipage}

	\begin{minipage}[b]{0.2\linewidth}
		\begin{center}
			\begin{circuitikz}[european voltages]
				%\draw (0,0) to [I, l^=$i(t)$] (0,2);
				\draw [-latex] (0.6, 1.85) -- (1.25, 1.85);
				\draw node[] at (0.25, 1.75) {$i(t)$};
				\draw (0,2) -- (1.5,2);
				\draw (0,0) -- (1.5,0);
				\draw (1.5,0) to [C, l^= $C$, v_=$e(t)$] (1.5,2);
			\end{circuitikz}
		\end{center}
	\end{minipage}
	\begin{minipage}[b]{0.7\linewidth}
		\begin{equation}\label{key}
		{e}(t) = \frac{1}{C}\int\limits_{ - \infty }^t {i(t)dt}
		\end{equation}
		\begin{equation}\label{key}
		i(t) = C\frac{de(t)}{dt}
		\end{equation}
	\end{minipage}

	\begin{minipage}[b]{0.2\linewidth}
		\begin{center}
			\begin{circuitikz}[european voltages]
				%\draw (0,0) to [I, l^=$i(t)$] (0,2);
				\draw [-latex] (0.6, 1.85) -- (1.25, 1.85);
				\draw node[] at (0.25, 1.75) {$i(t)$};
				\draw (0,2) -- (1.5,2);
				\draw (0,0) -- (1.5,0);
				\draw (1.5,0) to [L, l^= $L$, v_=$e(t)$] (1.5,2);
			\end{circuitikz}
		\end{center}
	\end{minipage}
	\begin{minipage}[b]{0.7\linewidth}
		\begin{equation}\label{key}
		e(t) = L\frac{{di(t)}}{{dt}}
		\end{equation}
		\begin{equation}\label{key}
		i(t) = \frac{1}{L}\int\limits_{ - \infty }^t {e(t)dt}
		\end{equation}
	\end{minipage}
\end{frame}

\begin{frame}
\frametitle{Recapitulação}
	Energia armazenada -> pag. 133.
\end{frame}
%------------------------------------------------------------------------------------
\section{Funções Singulares}
%------------------------------------------------------------------------------------
\begin{frame}
\frametitle{Funções Singulares}
\begin{itemize}
	\item Utilizados para representarem sinais em circuitos com operações chaveadas.
	\item Auxiliam na descrição de alguns fenômenos que ocorrem na análise de transitórios.
	\item Por definição, são funções descontínuas ou que possuem derivadas descontínuas.
	\item Funções singulares mais importante: \textbf{degrau}, \textbf{rampa} e \textbf{impulso}. 
\end{itemize}
\end{frame}

	
%------------------------------------------------
\subsection{Função Degrau}
\begin{frame}
	\frametitle{Sumário}
	\small
	\tableofcontents[currentsubsection]
\end{frame}
%%-----------------------------------------------
\begin{frame}
\frametitle{Funções Singulares}
\framesubtitle{Função Degrau}
	\begin{itemize}
		\item Utilizada para representar uma variação instantânea na tensão ou na corrente;
		\vfill
		\item Possíveis aplicações: circuito de controles e sistemas digitais.
	\end{itemize}
	
\end{frame}

\begin{frame}
\frametitle{Funções Singulares}
\framesubtitle{Função Degrau}
	\begin{minipage}[b]{0.45\linewidth}
		\begin{block}{Função degrau unitária}
			\begin{equation}\label{key}
			v(t) = \left\{ \begin{array}{l}
			0,~t < 0\\
			1,~t > 0
			\end{array} \right.
			\end{equation}
			Que é denotada por:
			\begin{equation}\label{key}
			v(t) = {U_{ - 1}}(t)
			\end{equation}
			\begin{center}
				\begin{circuitikz}					
					\begin{scope}[]
						\draw [-latex] (0,0) -- (3,0);
						\draw [-latex] (1.5,-0.5) -- (1.5,2);
						\draw node[] (F) at (3,-0.5) {$t$};
						\draw node[] (G) at (0.75,+2) {$U_{-1}(t)$};
						\draw [color=red] (0,0) -- (1.5,0);
						\draw [color=red] (1.5,0) -- (1.5,1);
						\draw [color=red] (1.5,1) -- (3,1);
						\draw node[] (I) at (1.25,1) {$1$};
					\end{scope}				
				\end{circuitikz}
			\end{center}
		\end{block}
	\end{minipage}
	\hfill
	\begin{minipage}[b]{0.45\linewidth}
		\begin{block}{Função degrau unitária inversa}
			\begin{equation}\label{key}
			v(t) = \left\{ \begin{array}{l}
			1,~t < 0\\
			0,~t > 0
			\end{array} \right.
			\end{equation}
			Que é denotada por:
			\begin{equation}\label{key}
			v(t) = {U_{ - 1}}(-t)
			\end{equation}
			\begin{center}
				\begin{circuitikz}					
					\begin{scope}[]
						\draw [-latex] (0,0) -- (3,0);
						\draw [-latex] (1.5,-0.5) -- (1.5,2);
						\draw node[] (F) at (3,-0.5) {$t$};
						\draw node[] (G) at (0.75,+2) {$U_{-1}(-t)$};
						\draw [color=red] (0,1) -- (1.5,1);
						\draw [color=red] (1.5,1) -- (1.5,0);
						\draw [color=red] (1.5,0) -- (2.85,0);
						\draw node[] (I) at (1.75,1) {$1$};
					\end{scope}				
				\end{circuitikz}
			\end{center}
		\end{block}
	\end{minipage}
\end{frame}

\begin{frame}
\frametitle{Funções Singulares}
\framesubtitle{Função Degrau}
\newcommand{\xshift}{6}
\newcommand{\yshift}{0}
\begin{overprint}
	\only<1>
	{
		\begin{center}
			\begin{circuitikz} 
				\draw node[] at (4, 4) {para $t<t_0$};
				\draw
				(0,2) to[V, l_=$V_0$] (0,0)
				(0,2) -- (1,2)
				node[circ] (A) at (1,2) {}
				node[circ] (B) at (1.5,2) {}
				node[circ] (C) at (1.25,1.5) {}
				(1.25,1.5) -- (1.25,0)
				(0,0) -- (1.25,0)
				node[circ] (D) at (2.5,2) {}
				node[circ] (E) at (2.5,0) {}
				(B) -- (D)
				(1.25,0) -- (E)
				(2,0) -- (2.5,0)
				(D) to[open, v^=$v(t)$] (E);
				\draw 
				[-latex] (B) -- (C);					
				
				\begin{scope}[]
					\draw [-latex] (\xshift-0.5,\yshift) -- (\xshift+3,\yshift);
					\draw [-latex] (\xshift,\yshift-0.5) -- (\xshift,\yshift+2);
					\draw node[] (F) at (\xshift+3,\yshift-0.5) {$t$};
					\draw node[] (G) at (\xshift-0.5,\yshift+2) {$v(t)$};
					\draw node[] (H) at (\xshift+1.5,\yshift-0.5) {$t_0$};
					\draw [color=red] (\xshift-0.5,\yshift) -- (\xshift+1.5,\yshift);
					\draw node[] (I) at (\xshift-0.5,\yshift+1) {$V_0$};
				\end{scope}				
			\end{circuitikz}
		\end{center}
	}	
	\only<2>
	{
		\begin{center}
			\begin{circuitikz} 
				\draw node[] at (4, 4) {para $t>t_0$};
				\draw
				(0,2) to[V, l_=$V_0$] (0,0)
				(0,2) -- (1,2)
				node[circ] (A) at (1,2) {}
				node[circ] (B) at (1.5,2) {}
				node[circ] (C) at (1.25,1.5) {}
				(1.25,1.5) -- (1.25,0)
				(0,0) -- (1.25,0)
				node[circ] (D) at (2.5,2) {}
				node[circ] (E) at (2.5,0) {}
				(B) -- (D)
				(1.25,0) -- (E)
				(2,0) -- (2.5,0)
				(D) to[open, v^=$v(t)$] (E);
				\draw [-latex] (B) -- (A);					
				
				\begin{scope}[]
					\draw [-latex] (\xshift-0.5,\yshift) -- (\xshift+3,\yshift);
					\draw [-latex] (\xshift,\yshift-0.5) -- (\xshift,\yshift+2);
					\draw node[] (F) at (\xshift+3,\yshift-0.5) {$t$};
					\draw node[] (G) at (\xshift-0.5,\yshift+2) {$v(t)$};
					\draw node[] (H) at (\xshift+1.5,\yshift-0.5) {$t_0$};
					\draw [color=red] (\xshift-0.5,\yshift) -- (\xshift+1.5,\yshift);
					\draw [color=red] (\xshift+1.5,\yshift) -- (\xshift+1.5,\yshift+1);
					\draw [color=red] (\xshift+1.5,\yshift+1) -- (\xshift+3,\yshift+1);
					\draw [dotted] (\xshift,\yshift+1) -- (\xshift+1.5,\yshift+1);
					\draw node[] (I) at (\xshift-0.5,\yshift+1) {$V_0$};
				\end{scope}				
			\end{circuitikz}
		\end{center}
	}	
	\only<3>
	{
		\begin{center}
			\begin{circuitikz} 
%				\draw node[] at (4, 2.5) {para $t>t_0$};
%				\draw
%				(0,2) to[V, l_=$V_0$] (0,0)
%				(0,2) -- (1,2)
%				node[circ] (A) at (1,2) {}
%				node[circ] (B) at (1.5,2) {}
%				node[circ] (C) at (1.25,1.5) {}
%				(1.25,1.5) -- (1.25,0)
%				(0,0) -- (1.25,0)
%				node[circ] (D) at (2.5,2) {}
%				node[circ] (E) at (2.5,0) {}
%				(B) -- (D)
%				(1.25,0) -- (E)
%				(2,0) -- (2.5,0)
%				(D) to[open, v^=$v(t)$] (E);
%				\draw [-latex] (B) -- (A);	
				\begin{scope}[]
					\draw [-latex] (\xshift-0.5,\yshift) -- (\xshift+3,\yshift);
					\draw [-latex] (\xshift,\yshift-0.5) -- (\xshift,\yshift+2);
					\draw node[] (F) at (\xshift+3,\yshift-0.5) {$t$};
					\draw node[] (G) at (\xshift-0.5,\yshift+2) {$v(t)$};
					\draw node[] (H) at (\xshift+1.5,\yshift-0.5) {$t_0$};
					\draw [color=red] (\xshift-0.5,\yshift) -- (\xshift+1.5,\yshift);
					\draw [color=red] (\xshift+1.5,\yshift) -- (\xshift+1.5,\yshift+1);
					\draw [color=red] (\xshift+1.5,\yshift+1) -- (\xshift+3,\yshift+1);
					\draw [dotted] (\xshift,\yshift+1) -- (\xshift+1.5,\yshift+1);
					\draw node[] (I) at (\xshift-0.5,\yshift+1) {$V_0$};
				\end{scope}				
			\end{circuitikz}
		\end{center}
		\begin{block}{Função degrau deslocada}
			\begin{equation}\label{key}
			v(t) = \left\{ \begin{array}{l}
			0,~t < {t_0}\\
			V_0,~t > {t_0}
			\end{array} \right.
			\end{equation}
			Ou, utilizando a função degrau deslocada:
			\begin{equation}\label{key}
			v(t) = {V_0}{U_{ - 1}}(t - {t_0})
			\end{equation}
		\end{block}
	}	
\end{overprint}
\end{frame}

\begin{frame}
\frametitle{Funções Singulares}
\framesubtitle{Função Degrau}
%	\begin{block}{Função degrau deslocada}
%		\begin{equation}\label{key}
%		v(t) = \left\{ \begin{array}{l}
%		0,~t < {t_0}\\
%		V_0,~t > {t_0}
%		\end{array} \right.
%		\end{equation}
%		Ou, utilizando a função degrau deslocada:
%		\begin{equation}\label{key}
%		v(t) = {V_0}{U_{ - 1}}(t - {t_0})
%		\end{equation}
%	\end{block}
	\begin{block}{Dica!}
		Para verificar em que posição do eixo $x$ o degrau ocorre, basta igualar o valor dentro dos parenteses a zero. Exemplo, em $v(t) = 2{U_{ - 1}}(t - 5)$, deve-se fazer $t - 5 = 0 \Rightarrow t = 5$.
	\end{block}	

	\begin{center}
		\begin{circuitikz} 
			%				\draw node[] at (4, 2.5) {para $t>t_0$};
			%				\draw
			%				(0,2) to[V, l_=$V_0$] (0,0)
			%				(0,2) -- (1,2)
			%				node[circ] (A) at (1,2) {}
			%				node[circ] (B) at (1.5,2) {}
			%				node[circ] (C) at (1.25,1.5) {}
			%				(1.25,1.5) -- (1.25,0)
			%				(0,0) -- (1.25,0)
			%				node[circ] (D) at (2.5,2) {}
			%				node[circ] (E) at (2.5,0) {}
			%				(B) -- (D)
			%				(1.25,0) -- (E)
			%				(2,0) -- (2.5,0)
			%				(D) to[open, v^=$v(t)$] (E);
			%				\draw [-latex] (B) -- (A);	
			\begin{scope}[]
				\draw [-latex] (-0.5,0) -- (+3,0);
				\draw [-latex] (0,-0.5) -- (0,+2);
				\draw node[] (F) at (+3,-0.5) {$t$};
				\draw node[] (G) at (-0.5,+2) {$v(t)$};
				\draw node[] (H) at (+1.5,-0.5) {$5$};
				\draw [color=red] (-0.5,0) -- (+1.5,0);
				\draw [color=red] (+1.5,0) -- (+1.5,+1);
				\draw [color=red] (+1.5,+1) -- (+3,+1);
				\draw [dotted] (0,+1) -- (+1.5,+1);
				\draw node[] (I) at (-0.5,+1) {$2$};
			\end{scope}				
		\end{circuitikz}
	\end{center}
	

\end{frame}
%------------------------------------------------
\subsection{Função Rampa}
\begin{frame}
\frametitle{Sumário}
\small
\tableofcontents[currentsubsection]
\end{frame}
%%-----------------------------------------------
\begin{frame}
\frametitle{Funções Singulares}
\framesubtitle{Função Rampa}
	\begin{minipage}[b]{0.45\linewidth}
		\begin{block}{Função rampa unitária}
			\vspace{0.25cm}
			\begin{equation}\label{key}			
			v(t) = \left\{ \begin{array}{l}
			0,~t < 0\\
			t,~t > 0
			\end{array} \right.
			\end{equation}
			Que é denotada por:
			\begin{equation}\label{key}
			v(t) = {U_{ - 2}}(t)
			\end{equation}
			\begin{center}
				\begin{circuitikz}	
					\coordinate (Y) at (3.5,2);
					\coordinate (A) at (1.5,0);
					\coordinate (X) at (4,0);				
					\begin{scope}[]
						\draw [-latex] (0,0) -- (3,0);
						\draw [-latex] (1.5,-0.5) -- (1.5,2);
						\draw node[] (F) at (3,-0.5) {$t$};
						\draw node[] (G) at (0.75,+2) {$U_{-2}(t)$};
						\draw [color=red] (0,0) -- (1.5,0);
						\draw [color=red] (1.5,0) -- (2.5,1);
					
						\path[clip] (A) -- (X) -- (Y);
						\fill[red, opacity=0.2, draw=black] (A) circle (5mm);
						\node at ($(A)+(22.5:9mm)$) {$45^\circ$};
					\end{scope}				
				\end{circuitikz}
			\end{center}
		\end{block}
	\end{minipage}
	\hfill
	\begin{minipage}[b]{0.45\linewidth}
		\begin{block}{Função rampa genérica}
			\begin{equation}\label{key}
			v(t) = \left\{ \begin{array}{l}
			0,~t < t_0\\
			At,~t > t_0
			\end{array} \right.
			\end{equation}
			Que é denotada por:
			\begin{equation}\label{key}
			v(t) = A{U_{ - 2}}(t-t_0)
			\end{equation}
			\begin{center}
				\begin{circuitikz}	
					\coordinate (Y) at (3.5,2);
					\coordinate (A) at (1.5,0);
					\coordinate (X) at (4,0);				
					\begin{scope}[]
						\draw [-latex] (0.5,0) -- (3,0);
						\draw [-latex] (1,-0.5) -- (1,1.5);
						\draw node[] (F) at (3,-0.5) {$t$};
						\draw node[] (G) at (0.5,1.5) {$v(t)$};
						\draw [color=red] (0.5,0) -- (1.5,0);
						\draw [color=red] (1.5,0) -- (2.5,1);
						\draw node[] (F) at (1.5,-0.5) {$t_0$};
						\draw node[] (F) at (2.5,1.5) {$tg(\alpha)=A$};
						
						\path[clip] (A) -- (X) -- (Y);
						\fill[red, opacity=0.2, draw=black] (A) circle (5mm);
						\node at ($(A)+(22.5:9mm)$) {$\alpha$};
					\end{scope}				
				\end{circuitikz}
			\end{center}
		\end{block}
	\end{minipage}
\end{frame}

\begin{frame}
\frametitle{Funções Singulares}
\framesubtitle{Função Rampa}
	\begin{block}{Relação entre função rampa e função degrau}
		Note que:
		\begin{equation}\label{key}
		{U_{ - 2}}(t) = \int\limits_{ - \infty }^t {{U_{ - 1}}(t)} dt
		\end{equation}
		Ou ainda:
		\begin{equation}\label{key}
		{U_{ - 1}}(t) = \frac{{d{U_{ - 2}}(t)}}{{dt}}
		\end{equation}
	\end{block}
\end{frame}


%------------------------------------------------
\subsection{Função Impulso}
\begin{frame}
\frametitle{Sumário}
\small
\tableofcontents[currentsubsection]
\end{frame}

\begin{frame}
\frametitle{Funções Singulares}
\framesubtitle{Função Impulso}
\begin{minipage}[b]{0.45\linewidth}
	\begin{block}{Função impulso unitário}
		\vspace{0.5cm}
		\begin{equation}\label{key}			
		v(t) = \left\{ \begin{array}{l}
		0,~t \neq 0\\
		1,~t = 0
		\end{array} \right.
		\end{equation}
		Que é denotada por:
		\begin{equation}\label{key}
		v(t) = {U_0}(t)
		\end{equation}
		\begin{center}
			\begin{circuitikz}									
				\begin{scope}[]
					\draw [-latex] (0,0) -- (3,0);
					\draw [-latex] (1.5,-0.5) -- (1.5,1.5);
					\draw node[] (F) at (3,-0.5) {$t$};
					\draw node[] (G) at (0.75,1.5) {$U_{0}(t)$};
					\draw [color=red] (0,0) -- (2.85,0);
					\draw [-latex, color=red] (1.5,0) -- (1.5,1);
					\draw node[] (G) at (1.25,1) {$1$};
				\end{scope}				
			\end{circuitikz}
		\end{center}
	\end{block}
\end{minipage}
\hfill
\begin{minipage}[b]{0.45\linewidth}
	\begin{block}{Função impulso genérico}
		%		\vspace{0.25cm}
		\begin{equation}\label{key}			
		v(t) = \left\{ \begin{array}{l}
		0,~t \neq t_0\\
		A,~t = t_0
		\end{array} \right.
		\end{equation}
		Que é denotada por:
		\begin{equation}\label{key}
		v(t) = A{U_0}(t-t_0)
		\end{equation}
		\begin{center}
			\begin{circuitikz}									
				\begin{scope}[]
					\draw [-latex] (0,0) -- (3,0);
					\draw [-latex] (1,-0.5) -- (1,1.5);
					\draw node[] (F) at (3,-0.5) {$t$};
					\draw node[] (G) at (-0.25,1.5) {$AU_{0}(t-t_0)$};
					\draw [color=red] (0,0) -- (2.85,0);
					\draw [-latex, color=red] (2,0) -- (2,1);
					\draw node[] (G) at (0.75,1) {$A$};
					\draw [dotted] (1,1) -- (2,1);
					\draw node[] (F) at (2,-0.5) {$t_0$};
				\end{scope}				
			\end{circuitikz}
		\end{center}
	\end{block}
\end{minipage}
\end{frame}

\begin{frame}
	\frametitle{Funções Singulares}
	\framesubtitle{Função Rampa}
	\begin{block}{Relação entre função degrau e função impulso}
		Note que:
		\begin{equation}\label{key}
		{U_{ - 1}}(t) = \int\limits_{ - \infty }^t {{U_{0}}(t)} dt
		\end{equation}
		Ou ainda:
		\begin{equation}\label{key}
		{U_{0}}(t) = \frac{{d{U_{ - 1}}(t)}}{{dt}}
		\end{equation}
	\end{block}
\end{frame}
%%-----------------------------------------------


%------------------------------------------------
\section{Resposta às Funções Singulares}
\begin{frame}
\frametitle{Sumário}
\small
\tableofcontents[currentsection]
\end{frame}

\begin{frame}
\frametitle{Resposta às Funções Singulares}
	\begin{theorem}
		Se todas as correntes e tensões permanecerem finitas, a tensão nos terminais de uma capacitância e a corrente passando por uma indutância não podem alterar instantaneamente.
	\end{theorem}
	\begin{itemize}
		\item Tal teorema é válido para circuitos elétricos excitados pelas funções \textbf{degrau} e \textbf{rampa}.
		\item A adição de uma função \textbf{impulso} faz com que considerações adicionais sejam observadas.
	\end{itemize}
	\begin{theorem}
		Um impulso unitário de corrente entrando em uma capacitância descarregada altera a sua tensão de $\frac{1}{C}V$ e a energia de $\frac{1}{2C}J$ instantaneamente.
	\end{theorem}
\end{frame}

\begin{frame}
\frametitle{Resposta às Funções Singulares}
\newcommand{\xshift}{4}
\newcommand{\yshift}{0.3}
\begin{overprint}
	\only<1>
	{
		\begin{theorem}[Demonstração]
			\begin{center}
				\begin{circuitikz}[european voltages]
					\draw (0,0) to [I, l^=$U_0(t)~A$] (0,2);
					\draw (0,2) -- (2,2);
					\draw (0,0) -- (2,0);
					\draw (2,0) to [C, l^= $C$, v_=$e_c(t)$] (2,2);
					\begin{scope}[]
						\draw [-latex] (0+\xshift,0+\yshift) -- (3+\xshift,0+\yshift);
						\draw [-latex] (1.5+\xshift,-0.2+\yshift) -- (1.5+\xshift,1.5+\yshift);
						\draw node[] (F) at (3+\xshift,0.25+\yshift) {$t$};
						\draw node[] (G) at (0.75+\xshift,1.5+\yshift) {$U_{0}(t)$};
						\draw [color=red] (0+\xshift,0+\yshift) -- (2.85+\xshift,0+\yshift);
						\draw [-latex, color=red] (1.5+\xshift,0+\yshift) -- (1.5+\xshift,1+\yshift);
						\draw node[] (G) at (1.25+\xshift,1+\yshift) {$1$};
					\end{scope}
				\end{circuitikz}
			\end{center}
			\begin{align}\label{key}
			{e_c}(t) = \frac{1}{C}\int\limits_{ - \infty }^t {i(t)dt = } \frac{1}{C}\int\limits_{ - \infty }^{{0^ - }} {{U_0}(t)dt + } \frac{1}{C}\int\limits_{{0^ - }}^{{0^ + }} {{U_0}(t)dt + } \frac{1}{C}\int\limits_{{0^ + }}^t {{U_0}(t)dt} 
			\end{align}
		\end{theorem}
	}
	\only<2>
	{
		\begin{theorem}[Demonstração]
			\begin{center}
				\begin{circuitikz}[european voltages]
					\draw (0,0) to [I, l^=$U_0(t)~A$] (0,2);
					\draw (0,2) -- (2,2);
					\draw (0,0) -- (2,0);
					\draw (2,0) to [C, l^= $C$, v_=$e_c(t)$] (2,2);
					\begin{scope}[]
						\draw [-latex] (0+\xshift,0+\yshift) -- (3+\xshift,0+\yshift);
						\draw [-latex] (1.5+\xshift,-0.2+\yshift) -- (1.5+\xshift,1.5+\yshift);
						\draw node[] (F) at (3+\xshift,0.25+\yshift) {$t$};
						\draw node[] (G) at (0.75+\xshift,1.5+\yshift) {$U_{0}(t)$};
						\draw [color=red] (0+\xshift,0+\yshift) -- (2.85+\xshift,0+\yshift);
						\draw [-latex, color=red] (1.5+\xshift,0+\yshift) -- (1.5+\xshift,1+\yshift);
						\draw node[] (G) at (1.25+\xshift,1+\yshift) {$1$};
					\end{scope}
				\end{circuitikz}
			\end{center}			
			\begin{align}\label{key}
			{e_c}(t) = \frac{1}{C}\int\limits_{ - \infty }^t {i(t)dt = } \frac{1}{C}\int\limits_{ - \infty }^{{0^ - }} {{U_0}(t)dt + } \frac{1}{C}\int\limits_{{0^ - }}^{{0^ + }} {{U_0}(t)dt + } \textcolor{red}{\frac{1}{C}\int\limits_{{0^ + }}^t {{U_0}(t)dt}} 
			\end{align}
			\textcolor{red}{Componente nula, já que o impulso ocorreu anteriormente.}
		\end{theorem}	
	}
	\only<3>
	{
		\begin{theorem}[Demonstração]
			\begin{center}
				\begin{circuitikz}[european voltages]
					\draw (0,0) to [I, l^=$U_0(t)~A$] (0,2);
					\draw (0,2) -- (2,2);
					\draw (0,0) -- (2,0);
					\draw (2,0) to [C, l^= $C$, v_=$e_c(t)$] (2,2);
					\begin{scope}[]
						\draw [-latex] (0+\xshift,0+\yshift) -- (3+\xshift,0+\yshift);
						\draw [-latex] (1.5+\xshift,-0.2+\yshift) -- (1.5+\xshift,1.5+\yshift);
						\draw node[] (F) at (3+\xshift,0.25+\yshift) {$t$};
						\draw node[] (G) at (0.75+\xshift,1.5+\yshift) {$U_{0}(t)$};
						\draw [color=red] (0+\xshift,0+\yshift) -- (2.85+\xshift,0+\yshift);
						\draw [-latex, color=red] (1.5+\xshift,0+\yshift) -- (1.5+\xshift,1+\yshift);
						\draw node[] (G) at (1.25+\xshift,1+\yshift) {$1$};
					\end{scope}
				\end{circuitikz}
			\end{center}			
			\begin{align}\label{key}
			{e_c}(t) = \frac{1}{C}\int\limits_{ - \infty }^t {i(t)dt = } \textcolor{red}{\frac{1}{C}\int\limits_{ - \infty }^{{0^ - }}
			{{U_0}(t)dt}} + \frac{1}{C}\int\limits_{{0^ - }}^{{0^ + }} {{U_0}(t)dt + }  
			0
			\end{align}
			\textcolor{red}{Componente que representa a energia inicial armazenada do capacitor. Componente nula se o capacitor for considerado inicialmente descarregado.}
		\end{theorem}	
	}
	\only<4>
	{
		\begin{theorem}[Demonstração]
			\begin{center}
				\begin{circuitikz}[european voltages]
					\draw (0,0) to [I, l^=$U_0(t)~A$] (0,2);
					\draw (0,2) -- (2,2);
					\draw (0,0) -- (2,0);
					\draw (2,0) to [C, l^= $C$, v_=$e_c(t)$] (2,2);
					\begin{scope}[]
						\draw [-latex] (0+\xshift,0+\yshift) -- (3+\xshift,0+\yshift);
						\draw [-latex] (1.5+\xshift,-0.2+\yshift) -- (1.5+\xshift,1.5+\yshift);
						\draw node[] (F) at (3+\xshift,0.25+\yshift) {$t$};
						\draw node[] (G) at (0.75+\xshift,1.5+\yshift) {$U_{0}(t)$};
						\draw [color=red] (0+\xshift,0+\yshift) -- (2.85+\xshift,0+\yshift);
						\draw [-latex, color=red] (1.5+\xshift,0+\yshift) -- (1.5+\xshift,1+\yshift);
						\draw node[] (G) at (1.25+\xshift,1+\yshift) {$1$};
					\end{scope}
				\end{circuitikz}
			\end{center}			
			\begin{equation}\label{key}
			{e_c}(t) = \frac{1}{C}\int\limits_{ - \infty }^t {i(t)dt = } 0 + \frac{1}{C}\int\limits_{{0^ - }}^{{0^ + }} {{U_0}(t)dt + } 0 = \frac{1}{C}{U_{ - 1}}(t) = 
			\textcolor{red}{\frac{1}{C}~V}
			\end{equation}
			e
			\begin{equation}\label{key}
			W(t) = \frac{1}{2}Ce_c^2(t) = \frac{1}{2}C{\left( {\frac{1}{C}{U_{ - 1}}(t)} \right)^2} = \frac{1}{{2C}}{U_{ - 1}}(t)
			= \textcolor{red}{\frac{1}{{2C}}~J}
			\end{equation}
			\textcolor{red}{Para $t>0$.}
		\end{theorem}			
	}
\end{overprint}
\end{frame}
%------------------------------------------------
\begin{frame}[noframenumbering]
	\vfill
	\centering
	\Huge{Thank you!}
	\vfill
\end{frame}

\end{document}
